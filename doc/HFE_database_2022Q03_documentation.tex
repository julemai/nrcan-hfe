\documentclass[10pt,a4paper,titlepage,parskip]{scrartcl}
\usepackage[utf8]{inputenc}
\usepackage{pdfpages}
\usepackage[english]{babel}
\renewcommand\familydefault{\sfdefault}
\usepackage{color}
\usepackage{etoolbox}
\usepackage{lastpage}
\usepackage{hyperref}
\usepackage{framed}
\usepackage{subcaption}

% adjust the page margins
\usepackage[left=2.5cm,right=2.5cm,top=2.5cm,bottom=3.0cm]{geometry}

\usepackage{fancyhdr}
\pagestyle{fancy}

\definecolor{myBlue}{rgb}{0.2,0.4,0.65}
\definecolor{ufzgray1}{RGB}{81,81,81}
\definecolor{ufzgray2}{RGB}{156,156,156}
\definecolor{ufzgray3}{RGB}{185,185,185}

\patchcmd{\headrule}{\hrule}{\color{myBlue}\hrule}{}{}
\patchcmd{\footrule}{\hrule}{\color{myBlue}\hrule}{}{}

% dots after section numbers
\renewcommand{\thesection}{\arabic{section}.}
\renewcommand{\thesubsection}{\arabic{section}.\arabic{subsection}.}

% make an entire region tt font
\newenvironment{ttfont}{\fontfamily{\ttdefault}\selectfont}{\par}

% My style
\usepackage{xcolor}
\definecolor{jmcolor}{RGB}{255,127,0}
\newcommand{\jm}[1]{\textcolor{jmcolor}{#1}}
\newcommand{\jmout}[1]{\textcolor{jmcolor}{\sout{#1}}}
\newcommand{\GRAU}[1]{\textcolor{ufzgray2}{#1}}

%\newcommand{\footrulecolor}{myBlue}
\rhead{}
\chead{\textcolor{myBlue}{\footnotesize Project Documentation ``Improving characterization of flood-producing precipitation events in HFE database''}}
\lfoot{\textcolor{myBlue}{\footnotesize Telephone: +1 (226) 505-5600}}
\cfoot{\textcolor{myBlue}{\footnotesize e-mail: juliane.mai@uwaterloo.ca}}
\rfoot{\textcolor{myBlue}{\footnotesize Page \thepage \ of \pageref{LastPage}}}
\renewcommand{\footrulewidth}{0.4pt} 

% Hurenkinder und Schusterjungen verhindern
\clubpenalty10000
\widowpenalty10000
\displaywidowpenalty=10000

\includepdfset{pages=-}

% colored lines
\makeatletter
\let\old@rule\@rule
\def\@rule[#1]#2#3{\textcolor{rulecolor}{\old@rule[#1]{#2}{#3}}}
\makeatother

% fixed width columns in table
\usepackage{array}
\newcolumntype{L}[1]{>{\raggedright\let\newline\\\arraybackslash\hspace{0pt}}p{#1}}
\newcolumntype{C}[1]{>{\centering\let\newline\\\arraybackslash\hspace{0pt}}p{#1}}
\newcolumntype{R}[1]{>{\raggedleft\let\newline\\\arraybackslash\hspace{0pt}}p{#1}}

%\linespread{1.5}

\begin{document}  
	
	\vspace*{-1cm}
	\pagestyle{fancy}
	
	\begin{center}
		Juliane Mai, Ph.D.\\
		Technical Consulting\\[4pt]
		405-460A Belmont Ave W., \\
		Kitchener, ON, N2M 0A9, Canada\\[10pt]
		{\definecolor{rulecolor}{named}{myBlue}\rule{\linewidth}{0.4pt}}
	\end{center}
	\begin{center}
		\textcolor{myBlue}{{\Large --~Documentation~--}}\\[4pt]
		Improving the characterization of the flood-producing precipitation events in the Historic Flood Event (HFE) database by using the CaPA reanalysis/analysis\\[4pt]
	\end{center}
	

\vspace*{0cm}

\section{Meetings}

\subsection{Kick-off Meeting}
\vspace*{-0.5cm}
\begin{tabbing}
	\hspace{2.8cm} \= \hspace{4cm} \= \kill \\
	\textbf{Date and time:} 	\> Aug 25, 2022 1-2pm\\
	\textbf{Attendees:} 	\> Dikra Khedhaouiria \> (dikraa.khedhaouiria@ec.gc.ca)\\
							\> Juliane Mai \> (juliane.mai@uwaterloo.ca)\\
							\> Marie-Eve Martin \> (marie-eve.martin@nrcan-rncan.gc.ca)\\
							\> Philippe Aussant \> (philippe.aussant@NRCan-RNCan.gc.ca)\\
	                        \> Pierre-Etienne Bonhomme \> (etienne.bonhomme@NRCan-RNCan.gc.ca)\\
	                        \> Vincent Fortin \> (vincent.fortin@ec.gc.ca)
\end{tabbing}
\vspace*{-0.5cm}
\textbf{Notes:}
\begin{itemize}
	\setlength\itemsep{-0.8em}
	\item discussion of access to resources (HFE database, Geomet requests, CaSPAr requests)
	\item discussion of formats (JSON, PNG, i.e., mimicking formats that are received when requesting data from GeoMet)
	\item discussion of handling liquid/solid precipitation (Vincent suggested manuscript in preparation that will require humidity and temperature as additional variables to transform precipitation into components)
	\item next meeting mid September will be scheduled by Dikra
\end{itemize}

\section{Resources}

\begin{itemize}
	\setlength\itemsep{-0.8em}
	\item access to the HFE database:\\
	\url{https://open.canada.ca/data/en/dataset/fe83a604-aa5a-4e46-903c-685f8b0cc33c}
	\item query to obtain RDPA 24h total precipitation for a 1x1 degree box centered on Montreal:\\
	\url{https://api.weather.gc.ca/collections/weather:rdpa:10km:24f/coverage?f=json&datetime=2017-05-12T12Z&CRS=EPSG:4326&bbox=-74,45,-73,46}
	\item documentation of requests from GeoMet:\\
	\url{https://eccc-msc.github.io/open-data/msc-geomet/web-services_en/}\\
	see specifically OGC API:\\
	\url{https://eccc-msc.github.io/open-data/msc-geomet/web-services_en/#ogc-api-features}
\end{itemize}


\section{Obtain data from GeoMet}

The assumption is that we want to retrieve CaPA data (RDPA; 24h total precipitation) from GeoMet for Aug 24, 2022 around Montreal ($\mathrm{lat} = 45.5019^{\circ}\mathrm{N}$, $\mathrm{lon}=-73.5674^{\circ}\mathrm{E}$). 

The URL to request is the following:
\url{https://api.weather.gc.ca/collections/weather:rdpa:10km:24f/coverage?f=json&datetime=2022-08-24T12Z&CRS=EPSG:4326&bbox=-74,45,-73,46}

This returns the following JSON string (not geo-referenced):
\begin{framed}	
		\texttt{\{"type": "Coverage", "domain": \{"type": "Domain", "domainType": "Grid", "axes": \{"x": \{"start": -74.0, "stop": -73.0, "num": 15\}, "y": \{"start": 46.0, "stop": 45.0, "num": 16\}\}, "referencing": [\{"coordinates": ["x", "y"], "system": \{"type": "GeographicCRS", "id": "http://www.opengis.net/def/crs/OGC/1.3/CRS84"\}\}]\}, "parameters": \{"APCP": \{"type": "Parameter", "description": "Total precipitation [kg/(m\^{}2)]", "unit": \{"symbol": "[kg/(m\^{}2)]"\}, "observedProperty": \{"id": "0-SFC", "label": \{"en": "Total precipitation [kg/(m\^{}2)]"\}\}\}\}, "ranges": \{"APCP": \{"type": "NdArray", "dataType": "float", "axisNames": ["y", "x"], "shape": [16, 15], "values": [null, null, null, null, null, null, null, null, null, null, null, null, null, null, null, null, null, null, null, null, null, 9.300250053405762, 12.344250679016113, null, null, null, null, null, null, null, null, null, null, null, null, 5.607500076293945, 9.330750465393066, 13.382000923156738, 16.639751434326172, null, null, null, null, null, null, $\ldots$ , null, null, null, null, null, null, null, 10.574250221252441, 11.665250778198242, 11.032000541687012, null, null, null, null, null, null, null, null, null, null, null, null, 9.582500457763672, 10.17750072479248, null, null, null, null, null, null, null, null, null, null, null, null, null, null, null, null, null, null, null, null, null]\}\}\}}\\
\end{framed}

This corresponds to the following 2D view of the data:
\begin{table}[h]
	\begin{tabular}{R{0.5cm}R{0.5cm}R{0.5cm}R{0.5cm}R{0.5cm}R{0.5cm}R{0.5cm}R{0.5cm}R{0.5cm}R{0.5cm}R{0.5cm}R{0.5cm}R{0.5cm}R{0.5cm}R{0.5cm}}
- & - & - & - & - & - & - & - & - & - & - & - & - & - & -\\
- & - & - & - & - & - & 9 & 12 & - & - & - & - & - & - & -\\
- & - & - & - & - & 6 & 9 & 13 & 17 & - & - & - & - & - & -\\
- & - & - & - & 4 & 6 & 9 & 13 & 17 & 21 & - & - & - & - & -\\
- & - & 3 & 3 & 4 & 6 & 11 & 16 & 19 & 22 & - & - & - & - & -\\
- & 4 & 3 & 3 & 3 & 6 & 13 & 18 & 22 & 25 & 23 & - & - & - & -\\
- & 5 & 4 & 3 & 3 & 5 & 12 & 19 & 23 & 28 & 26 & 28 & - & - & -\\
- & - & 5 & 3 & 2 & 4 & 12 & 18 & 23 & 26 & 27 & 32 & 38 & - & -\\
- & - & - & 3 & 2 & 4 & 10 & 16 & 20 & 22 & 26 & 35 & 38 & 27 & -\\
- & - & - & - & 4 & 5 & 8 & 16 & 18 & 20 & 27 & 36 & 35 & 24 & -\\
- & - & - & - & 6 & 6 & 7 & 14 & 16 & 20 & 31 & 41 & 35 & 20 & -\\
- & - & - & - & - & 5 & 8 & 11 & 13 & 16 & 27 & 37 & 33 & - & -\\
- & - & - & - & - & - & 10 & 10 & 12 & 11 & 19 & 26 & - & - & -\\
- & - & - & - & - & - & - & 11 & 12 & 11 & - & - & - & - & -\\
- & - & - & - & - & - & - & 10 & 10 & - & - & - & - & - & -\\
- & - & - & - & - & - & - & - & - & - & - & - & - & - & -\\				
	\end{tabular}
\end{table}
\pagebreak

To retrieve the corresponding PNG file for this data use (see Fig.~\ref{fig:plot_example:geo-weather}):
\begin{framed}
\url{https://geo.weather.gc.ca/geomet?SERVICE=WMS&VERSION=1.3.0&REQUEST=GetMap&LAYERS=RDPA.24F_PR&STYLES=RDPA-WXO&CRS=EPSG:4326&BBOX=45,-74,46,-73&WIDTH=400&HEIGHT=400&FORMAT=image/png&TIME=2022-08-24T12:00:00Z&DIM_REFERENCE_TIME=2022-08-24T12:00:00Z}
\end{framed}
To retrieve the legend use (see Fig.~\ref{fig:plot_example:geo-weather}):
\begin{framed}
	\url{https://geo.weather.gc.ca/geomet?SERVICE=WMS&VERSION=1.3.0&REQUEST=GetLegendGraphic&LAYERS=RDPA.24F_PR&STYLES=RDPA-WXO&CRS=EPSG:4326&BBOX=45,-74,46,-73&WIDTH=400&HEIGHT=400&FORMAT=image/png&TIME=2022-08-24T12:00:00Z&DIM_REFERENCE_TIME=2022-08-24T12:00:00Z}
\end{framed}
\begin{figure}[h]
	\begin{subfigure}[b]{0.45\textwidth}
		\centering
		\includegraphics[height=0.6\linewidth]{figures/test-map-geo-weather.png}
		\hspace*{0.5cm}
		\includegraphics[height=0.6\linewidth]{figures/test-legend-geo-weather.png}
		\caption{PNG and legend retrieved using \url{https://geo.weather.gc.ca}\\(only displayed for reference)}
		\label{fig:plot_example:geo-weather}
	\end{subfigure}
	\hspace*{0.05\textwidth}
	\begin{subfigure}[b]{0.45\textwidth}
		\centering
		\includegraphics[height=0.6\linewidth]{figures/test-map-api-weather-nrcan-hfe.png}
		\hspace*{0.5cm}
		\includegraphics[height=0.6\linewidth]{figures/test-legend-api-weather-nrcan-hfe.png}
		\caption{PNG and legend produced using NRCan-HFE library based on GRIB2 data retrieved using \url{https://api.weather.gc.ca/}}
		\label{fig:plot_example:api-weather}
	\end{subfigure}
	\caption{RDPA (24h total precipitation) on Aug 24, 2022 (noon UTC) around Montreal}
	\label{fig:plot_example}
\end{figure}

To actually retrieve the geo-referenced data used for those plots, the GRIB2 file needs to be retrieved using the following command:
\begin{framed}
\texttt{\url{https://api.weather.gc.ca/collections/weather:rdpa:10km:24f/coverage?f=GRIB&datetime=2022-08-24T12Z&CRS=EPSG:4326&bbox=-74,45,-73,46}}
\end{framed}
This seems to be the only possibility to obtain geo-referenced data. The library developed here now provides three functions to obtain and plot the data. The results are shown in Fig.~\ref{fig:plot_example:api-weather}.

\textbf{Step A1:} Request data using the Geomet API (\url{https://api.weather.gc.ca}).
\begin{framed}
	\vspace*{-1.2cm}
	\begin{ttfont}
	\begin{tabbing}
		\hspace{1.0cm} \= \hspace{3.2cm} \= \kill \\[4pt]
		\GRAU{\# see module for detailed documentation and example}\\
		from a1\_request\_geomet\_grib2 import request\_geomet\_grib2\\
		\\
		files\_geomet = request\_geomet\_grib2(\\
		\> product=\textit{product}, \> \GRAU{\# name of product, e.g., ``rdpa:10km:24f''}\\
		\> date=\textit{date},\> \GRAU{\# datetime object specifying date (can be list of dates)}\\
		\> bbox=\textit{bbox},\> \GRAU{\# dictionary specifying bounding box}\\
		\> crs=\textit{crs},\> \GRAU{\# coordinate reference system, e.g., ``EPSG:4326''}\\
		\> filename=\textit{filename},\> \GRAU{\# base filename of output file (can include path but no}\\
		\>                           \> \GRAU{\# file extension); date will be added to filename}\\
		\> silent=\textit{silent}, \> \GRAU{\# True for no printing on screen}\\
		\> ) \> 
	\end{tabbing}
	\end{ttfont}
	\vspace*{-0.3cm}
\end{framed}
\vspace*{-0.3cm}
The return variable \texttt{files\_geomet} is a list of files requested and downloaded. If the file already existed, it will not be overwritten unless \texttt{overwrite} is set to \texttt{True}. The filename will be returned nonetheless. 

\textbf{Step A2:} Read data from files requested.
\begin{framed}
	\vspace*{-1.2cm}
	\begin{ttfont}
		\begin{tabbing}
			\hspace{1.0cm} \= \hspace{5.2cm} \= \kill \\[4pt]
			\GRAU{\# see module for detailed documentation and example}\\
			from a2\_read\_geomet\_grib2 import read\_geomet\_grib2\\
			\\
			data\_geomet = read\_geomet\_grib2(\\
			\> filenames=\textit{files\_geomet}, \> \GRAU{\# list of GRIB2 files to read}\\
			\> silent=\textit{silent}, \> \GRAU{\# True for no printing on screen}\\
			\> ) \> 
		\end{tabbing}
	\end{ttfont}
	\vspace*{-0.3cm}
\end{framed}
\vspace*{-0.3cm}
The returned variable \texttt{data\_geomet} is a dictionary that will contain the attributes \texttt{lat}, \texttt{lon}, and \texttt{var}. The latter will be 3-dimensional if several files were read. The latitude and longitudes of the files are checked for consistency.

\textbf{Step A3:} Plot data.
\begin{framed}
	\vspace*{-1.2cm}
	\begin{ttfont}
		\begin{tabbing}
			\hspace{1.0cm} \= \hspace{5.2cm} \= \kill \\[4pt]
			\GRAU{\# see module for detailed documentation and example}\\
			from a3\_plot\_geomet\_grib2 import plot\_geomet\_grib2\\
			\\
			plot\_geomet = plot\_geomet\_grib2(\\
			\> var=\textit{var}, \> \GRAU{\# 2D/3D array of values for variable}\\
			\> lat=\textit{lat}, \> \GRAU{\# 2D array of latitudes}\\
			\> lon=\textit{lon}, \> \GRAU{\# 2D array of longitudes}\\
			\> date=\textit{date}, \> \GRAU{\# date or list of dates as datetime objects}\\
			\> png=\textit{png}, \> \GRAU{\# True if PNG file(s) should be created}\\
			\> gif=\textit{gif}, \> \GRAU{\# True if GIF should be created}\\
			\> legend=\textit{legend}, \> \GRAU{\# True if PNG of legend should be created}\\
			\> cities=\textit{cities}, \> \GRAU{\# True to display cities on maps}\\
			\> basefilename=\textit{basefilename}, \> \GRAU{\# String specifying basename of files created}\\
			\> silent=\textit{silent}, \> \GRAU{\# True for no printing on screen}\\
			\> ) \> 
		\end{tabbing}
	\end{ttfont}
	\vspace*{-0.3cm}
\end{framed}
\vspace*{-0.3cm}
The returned variable \texttt{plot\_geomet} is a dictionary that will contain the attributes \texttt{png}, \texttt{gif}, and \texttt{legend}. Each of them are assigned list of the according files created. If no file was created (e.g., no legend), the list will be empty.

\newpage
\section{Time sheet}

Listed hours spend per day with a brief description of activities are provided in Table~\ref{tab:hours}.
\begin{table}[h]
	\begin{tabular}{L{2cm}R{1cm}L{11.5cm}}
		\hline
		\textbf{Date} & \textbf{Hours} & \textbf{Description}\\
		\hline
		Aug 25, 2022 & 4 & Kick-off meeting; collecting resources; discussion GeoMet requests; documentation \\
		Aug 26, 2022 & 3 & discussion GeoMet requests; documentation \\
		Aug 27, 2022 & 12 & coding Geomet functions (a1, a2, a3); documentation \\		  
		\hline
	\end{tabular}
	\caption{Project time sheet for Juliane Mai}
	\label{tab:hours}
\end{table}

\end{document}



